\documentclass{article}

\usepackage[utf8]{inputenc}
\usepackage[spanish]{babel}
\usepackage{multicol}
\usepackage[margin={26mm, 22mm}]{geometry}
\usepackage{amsmath}
\usepackage{amssymb}

{\renewcommand{\arraystretch}{1.5}

\title{Ampliación de Matemáticas 2: Formulario\footnote{\texttt{https://github.com/garciparedes/amat2-cheatsheet}}}
\author{Sergio García Prado \\ \texttt{sergio@garciparedes.me}}

\begin{document}

  \pagenumbering{gobble}

  \maketitle

  \begin{multicols}{2}


    \section{Cambio de Variable}

      \paragraph{Jacobiano}

      \begin{align*}
        \left|J\right| &= abs\left(det\left(
          \begin{array}{ccc}
            \frac{\partial x}{\partial u} & \frac{\partial x}{\partial v} & \frac{\partial x}{\partial w} \\
            \frac{\partial y}{\partial u} & \frac{\partial y}{\partial v} & \frac{\partial y}{\partial w} \\
            \frac{\partial z}{\partial u} & \frac{\partial z}{\partial v} & \frac{\partial z}{\partial w}
          \end{array}
        \right)\right)
      \end{align*}


      \paragraph{Coordenadas Esféricas}

        \begin{align*}
          T &:
          \begin{cases}
            x &= r \cdot sen(\alpha) \cdot cos(\beta) \\
            y &= r \cdot sen(\alpha) \cdot sen(\beta) \\
            z &= r \cdot cos(\alpha) \\
          \end{cases} \\
          \left|J\right| &= r ^ 2 \cdot sen(\alpha)
        \end{align*}

      \paragraph{Coordenadas Cilíndricas}

        \begin{align*}
          T &:
          \begin{cases}
            z &= r \cdot cos(\alpha)\\
            y &= r \cdot sin(\alpha)\\
            z &= z \\
          \end{cases} \\
          \left|J\right| &= r
        \end{align*}

    \section{Integral Impropia}

      \paragraph{Criterios de Comparación}
        \begin{align*}
          B &:= \text{Interior de bola de radio 1 centrada en 0} \\
          B^* &:= \text{Exterior de bola de radio 1 centrada en 0} \\
          r &= \sqrt{x_1^2 + ... + x_n^2}
        \end{align*}
        \begin{align*}
          \int_B r^{-\alpha} dx_1...dx_n &
          \begin{cases}
            \alpha < n & convergente\\
            \alpha \geq n & divergente
          \end{cases}
        \end{align*}

        \begin{align*}
          \int_{B^*}r^{-\alpha} dx_1...dx_n &
          \begin{cases}
            \alpha > n & convergente\\
            \alpha \leq n & divergente
          \end{cases}
        \end{align*}
    \section{Ecuación en Diferencias}
  \end{multicols}

\end{document}
